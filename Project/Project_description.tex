%Bernardo

\subsection{TinyOS}
The primary aim of the Project is to understand if two motes are too close each other; 
in order to obtain this, we have thought to exploit the \textit{Receive Signal Strengh Indication (RSSI)}
from the packets received. \\
Therefore, we include in the \textbf{ProjectAppC.nc} file all the components to extract the power received 
from the packet according to the specific radio used. \\
The temporal accuracy of the simulation was \textit{500 ms}, so each mote will send a packet in broadcast 
at the expiration of a timer, that happens every \textit{500 ms}.\\
The problem was then that if an \textit{Alarm} was solicited, the very next \textit{Alarm} should have been
relaunched at least after 1 minute (to avoid useless repetition of warnings).
So, we have created in the \textbf{ProjectC.nc} file a structure called \textit{moteMemory[NUMBER\_OF\_MOTES][2]}: for each mote, 
this structure keeps memory of the time passed between the last \textit{Alarm} with every other mote. 
The structure is initialized in the function \textbf{AMControl.startDone} with all zeros; as we can easily understand, it has as
many rows as thew number of motes we want to consider in the simulation and two columns:
\begin{itemize}
    \item The first column is a boolean, in particular the first column of row \textit{i} is 1 at mote \textit{j} if mote 
            \textit{j} has received a message from mote with ID = \textit{i} less than a minute ago.
    \item The second column has a value different from 0 only if the first column of the same row is 1; in this case, it indicates
            the time elapsed from the last notification sent in multiple of the timer duration (so \textit{500 ms}). Thus, this
            column of the row \textit{i} is updated when the timer fires and the first column of the same row is 1.
\end{itemize}
At this point, we only need to define how it is updated the first column of each row. This is done looking at the \textit{RSSI} of 
a received packet from mote \textit{i}: if it's higher than the \textit{POWER\_THRESHOLD} and if the boolean value of the row
\textit{i} of the matrix is 0 (meaning that it has passed at least one minute from the last \textit{Alarm} sent), we show off the
motes that start the event and the RSSI, and send a message to the NodeRed application.


\subsection{NodeRed}

In the \textbf{NodeRed} application, we first connect to the \textbf{Cooja} simulation; this can be done using the \textit{TCP\_IN} 
block in which we listen for the TCP requests at the default ports (60001, 60002, 60003, ...); 
we have put up to 14 connections, but this can be easily upgraded increasing the number of blocks.\\ 
Then, the payload of a received packet is filtered in order to retrieve correctly the information about: 
\begin{itemize}
    \item The source and the destination motes, the ones that have triggered the the alarm since too close.
    \item The power of the message exchanged by them (that is a function of the distance between the two motes).
\end{itemize}
Subsequently, we adapt the format of the payload to be suitable for the IFTTT syntax.\\
Finally we send an HTTPS request to the IFTTT account specified, that will trigger a command to send an email with the 
data of the \textit{Alarm}.