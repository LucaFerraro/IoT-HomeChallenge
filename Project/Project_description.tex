%Bernardo

\subsection{TinyOS}

\subsection{NodeRed}

In the \textbf{NodeRed} application we first need to connect to the Cooja simulation;
this can be done using the TCP_IN block in which we listen for the TCP requests
at the default ports (\textit{60001}, \textit{60002}, \textit{60003}, ...).
In the NodeRed we put up to 14 connections but this can be easily upgraded
increasing the number of blocks.
Then the payload is filtered in order to retrieve correctly the information 
about the Src and Dst motes that triggered the \textit{Alarm} and the power of the message (the 
distance between the two motes).
Subsequently we adapt the format of the payload in order to be suitable for the 
IFTTT syntax.
Finally we send a HTTPS request to the IFTTT account specified, that will trigger a command
to send an Email with the specific data of the \textit{Alarm}.

