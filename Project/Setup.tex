%Luca
In order to locally test this project, some requirements need to be met.
The first thing is to register to \textbf{IFTTT} and have a valid account since in this project, when two nodes
are too close one to the other, a notification is sent via email through this application.
Once logged into the account, it's necessary to create an Applet for the notification system following those steps:
\begin{itemize}
    \item Create a new applet.
    \item In the section \textbf{This}, select the voice \textit{Webhooks} to trigger the reception of a web request.
    \item Name the event \textit{Proximity\_alarm}; the name is fundamental for the proper working of the application.
    \item In the section \textbf{That}, choose \textit{Email} and insert the email where you want to receive notifications.
\end{itemize}
Now the actual simulation can start. \\

First of all, we need to compile the code to produce an executable file. To do this, we need to navigate from the terminal
to the directory of the project and run the command \texttt{make telosb}: the result is the creation of a folder (named 
\textbf{build}) containing the executable file that \textbf{Cooja}, the simulation framework, will run inside the motes.\\ 

The next step is to open \textbf{Cooja}, create a new simulation and add as \textit{Skymotes} the number of motes we 
want to use in the simulation. Each mote will correspond to a device and will be able to send notifications.\\ 
Pay attention that the number of nodes can't be greater than the parameter \textbf{NUMBER\_OF\_MOTES}
that is specified in the file \texttt{Project.h} (the default value is 5 but it can be easily changed before the creation of the executable file).\\ 
As an additional constraint, the maximum number of allowed motes cannot be greater than 14 which is the number of TCP connections 
managed by the \textbf{NodeRed} application.\\ 
In case the simulation would require a greater number of motes, the file \texttt{flows.json} must be changed adding the needed \textit{tcp in} blocks.\\ 
Once added the motes, it's necessary to start a socket on each of them (which will act as a server), on ports from 60001 to 
60014, if the file \texttt{flows.json} has not been changed from the user, or any other port inserted in the \textbf{NodeRed} application.\\ 

At this point, we will need to start \textbf{NodeRed} and to do this we will need to:
\begin{itemize}
    \item Open the terminal.
    \item Type the command \texttt{node-red}.
    \item Open a browser and connect to the local host (\textit{http://127.0.0.1:1880/}).
    \item Import the file \texttt{flows.json}.
    \item Deploy the flow (check if the tcp blocks corresponding to the port on which the socket has been opened in 
            Cooja is actually connected to that socket). 
\end{itemize}
Note that, by default, the block \textit{Req Params} is connected to the block \textit{Web Req}, that sends a mail through
\textbf{IFTTT} to one of our mail. In order to let a user receive the notification as a mail, the block \textit{Req Params} must
instead be connected to the block \textit{INSERT YOUR IFTTT KEY TO GET EMAILS}, which must be filled with the key of the mail
registered in \textbf{IFTTT}. (All this is also explained inside the block \textit{IFTTT instructions} of the NodeRed flow.)\\ 

Now everything is ready to start the simulation: just go again in Cooja and click on the \textit{Start} button. By moving the 
motes around, it's possible to see the messages exchanged by the motes both in Cooja and in NodeRed. If two motes get closer 
than a given threshold, a notification will be sent to the inserted email address.